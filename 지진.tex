%	-------------------------------------------------------------------------------
% 
%
%
%
%
%
%
%
%
%
%	-------------------------------------------------------------------------------

	\documentclass[12pt, a4paper, oneside]{book}
%	\documentclass[12pt, a4paper, landscape, oneside]{book}

		% --------------------------------- 페이지 스타일 지정
		\usepackage{geometry}
%		\geometry{landscape=true	}
		\geometry{top 		=10em}
		\geometry{bottom		=10em}
		\geometry{left		=8em}
		\geometry{right		=8em}
		\geometry{headheight	=4em} % 머리말 설치 높이
		\geometry{headsep		=2em} % 머리말의 본문과의 띠우기 크기
		\geometry{footskip		=4em} % 꼬리말의 본문과의 띠우기 크기
% 		\geometry{showframe}
	
%		paperwidth 	= left + width + right (1)
%		paperheight 	= top + height + bottom (2)
%		width 		= textwidth (+ marginparsep + marginparwidth) (3)
%		height 		= textheight (+ headheight + headsep + footskip) (4)



		%	===================================================================
		%	package
		%	===================================================================
%			\usepackage[hangul]{kotex}				% 한글 사용
			\usepackage{kotex}						% 한글 사용
			\usepackage[unicode]{hyperref}			% 한글 하이퍼링크 사용
			\usepackage{amssymb,amsfonts,amsmath}	% 수학 수식 사용

			\usepackage{scrextend}					% 
		
			\usepackage{enumerate}			%
			\usepackage{enumitem}			%
			\usepackage{tablists}			%	수학문제의 보기 등을 표현하는데 사용
										%	tabenum


		% ------------------------------ table 
			\usepackage{longtable}			%
			\usepackage{tabularx}			%

			\usepackage{setspace}			%
			\usepackage{booktabs}			% table
			\usepackage{color}				%
			\usepackage{multirow}			%
			\usepackage{boxedminipage}		% 미니 페이지
			\usepackage[pdftex]{graphicx}	% 그림 사용
			\usepackage[final]{pdfpages}	% pdf 사용
			\usepackage{framed}			% pdf 사용
			
			\usepackage{fix-cm}	
			\usepackage[english]{babel}
	
			\usepackage{tikz}%
			\usetikzlibrary{arrows,positioning,shapes}
			%\usetikzlibrary{positioning}
			

		% --------------------------------- 	page
			\usepackage{afterpage}			% 다음페이지가 나온면 어떻게 하라는 명령 정의 패키지
%			\usepackage{fullpage}			% 잘못 사용하면 다 흐트러짐 주의해서 사용
%			\usepackage{pdflscape}			% 
			\usepackage{lscape}			%	 


			\usepackage{blindtext}
	
		% --------------------------------- font 사용
			\usepackage{pifont}				%
			\usepackage{textcomp}
			\usepackage{gensymb}
			\usepackage{marvosym}






		% --------------------------------- 페이지 스타일 지정

		\usepackage[Sonny]		{fncychap}

			\makeatletter
			\ChNameVar	{\Large\bf}
			\ChNumVar		{\Huge\bf}
			\ChTitleVar	{\Large\bf}
			\ChRuleWidth	{0.5pt}
			\makeatother

%		\usepackage[Lenny]		{fncychap}
%		\usepackage[Glenn]		{fncychap}
%		\usepackage[Conny]		{fncychap}
%		\usepackage[Rejne]		{fncychap}
%		\usepackage[Bjarne]	{fncychap}
%		\usepackage[Bjornstrup]{fncychap}

		\usepackage{fancyhdr}
		\pagestyle{fancy}
		\fancyhead{} % clear all fields
		\fancyhead[LO]{\footnotesize \leftmark}
		\fancyhead[RE]{\footnotesize \leftmark}
		\fancyfoot{} % clear all fields
		\fancyfoot[LE,RO]{\large \thepage}
		%\fancyfoot[CO,CE]{\empty}
		\renewcommand{\headrulewidth}{1.0pt}
		\renewcommand{\footrulewidth}{0.4pt}
	
	
	
		% --------------------------------- 	section 스타일 지정
	
		\usepackage{titlesec}
		
		\titleformat*{\section}			{\large\bfseries}
		\titleformat*{\subsection}			{\normalsize\bfseries}
		\titleformat*{\subsubsection}		{\normalsize\bfseries}
		\titleformat*{\paragraph}			{\normalsize\bfseries}
		\titleformat*{\subparagraph}		{\normalsize\bfseries}
	
		\renewcommand{\thesection}			{\arabic{section}.}
		\renewcommand{\thesubsection}		{\thesection\arabic{subsection}.}
		\renewcommand{\thesubsubsection}	{\thesubsection\arabic{subsubsection}}
		
		\titlespacing*{\section} 			{0pt}{1.0em}{1.0em}
		\titlespacing*{\subsection}	  		{0ex}{1.0em}{1.0em}
		\titlespacing*{\subsubsection}		{0ex}{1.0em}{1.0em}
		\titlespacing*{\paragraph}			{0ex}{1.0em}{1.0em}
		\titlespacing*{\subparagraph}		{0ex}{1.0em}{1.0em}
	
	%	\titlespacing*{\section} 			{0pt}{0.0\baselineskip}{0.0\baselineskip}
	%	\titlespacing*{\subsection}	  		{0ex}{0.0\baselineskip}{0.0\baselineskip}
	%	\titlespacing*{\subsubsection}		{6ex}{0.0\baselineskip}{0.0\baselineskip}
	%	\titlespacing*{\paragraph}			{6pt}{0.0\baselineskip}{0.0\baselineskip}
	

		% --------------------------------- recommend		섹션별 페이지 상단 여백
		\newcommand{\SectionMargin}			{\newpage  \null \vskip 2cm}
		\newcommand{\SubSectionMargin}		{\newpage  \null \vskip 2cm}
		\newcommand{\SubSubSectionMargin}	{\newpage  \null \vskip 2cm}


	
		% --------------------------------- 장의 목차
		\usepackage{minitoc}
		\setcounter{minitocdepth}{1}    	% Show until subsubsections in minitoc
		\setlength{\mtcindent}{12pt} 		% default 24pt
	
	
		% --------------------------------- 	문서 기본 사항 설정
		\setcounter{secnumdepth}{3} 		% 문단 번호 깊이
		\setcounter{tocdepth}{3} 			% 문단 번호 깊이
		\setlength{\parindent}{0cm} 		% 문서 들여 쓰기를 하지 않는다.
		
		
		% --------------------------------- 	줄간격 설정
		\doublespace
%		\onehalfspace
%		\singlespace
		
		
% 	============================================================================== List global setting
%		\setlist{itemsep=1.0em}
	
% 	============================================================================== enumi setting

%		\renewcommand{\labelenumi}{\arabic{enumi}.} 
%		\renewcommand{\labelenumii}{\arabic{enumi}.\arabic{enumii}}
%		\renewcommand{\labelenumii}{(\arabic{enumii})}
%		\renewcommand{\labelenumiii}{\arabic{enumiii})}


	%	-------------------------------------------------------------------------------
	%		Vertical and Horizontal spacing
	%	-------------------------------------------------------------------------------
		\setlist[enumerate,1]	{ leftmargin=8.0em, rightmargin=0.0em, labelwidth=0.0em, labelsep=0.0em }
		\setlist[enumerate,2]	{ leftmargin=8.0em, rightmargin=0.0em, labelwidth=0.0em, labelsep=0.0em }
		\setlist[enumerate,3]	{ leftmargin=8.0em, rightmargin=0.0em, labelwidth=0.0em, labelsep=0.0em }
		\setlist[enumerate]	{ 	itemsep=1.0em, 
								leftmargin=6.0ex, 
								rightmargin=0.0em, 
								labelwidth=0.0em, 
								labelsep=4.0ex 
							}


	%	-------------------------------------------------------------------------------
	%		Label
	%	-------------------------------------------------------------------------------
%		\setlist[enumerate,1]{ label=\arabic*., ref=\arabic* }
%		\setlist[enumerate,1]{ label=\emph{\arabic*.}, ref=\emph{\arabic*} }
%		\setlist[enumerate,1]{ label=\textbf{\arabic*.}, ref=\textbf{\arabic*} }   	% 1.
%		\setlist[enumerate,1]{ label=\textbf{\arabic*)}, ref=\textbf{\arabic*)} }		% 1)
		\setlist[enumerate,1]{ label=\textbf{(\arabic*)}, ref=\textbf{(\arabic*)} }	% (1)
		\setlist[enumerate,2]{ label=\textbf{\arabic*)}, ref=\textbf{\arabic*)} }		% 1)
		\setlist[enumerate,3]{ label=\textbf{\arabic*.}, ref=\textbf{\arabic*.} }		% 1.

%		\setlist[enumerate,2]{ label=\emph{\alph*}),ref=\theenumi.\emph{\alph*} }
%		\setlist[enumerate,3]{ label=\roman*), ref=\theenumii.\roman* }


% 	============================================================================== itemi setting


	%	-------------------------------------------------------------------------------
	%		Vertical and Horizontal spacing
	%	-------------------------------------------------------------------------------
		\setlist[itemize]{itemsep=0.0em}


	%	-------------------------------------------------------------------------------
	%		Label
	%	-------------------------------------------------------------------------------
		\renewcommand{\labelitemi}{$\bullet$}
		\renewcommand{\labelitemii}{$\cdot$}
		\renewcommand{\labelitemiii}{$\diamond$}
		\renewcommand{\labelitemiv}{$\ast$}		




		% --------------------------------- recommend  글자 색깔지정 명령
		\newcommand{\red}		{\color{red}}			% 글자 색깔 지정
		\newcommand{\blue}		{\color{blue}}		% 글자 색깔 지정
		\newcommand{\black}	{\color{black}}		% 글자 색깔 지정
		\newcommand{\superscript}[1]{${}^{#1}$}

	
	
		% --------------------------------- 환경 정의 : 박스 치고 안의 글자 빨간색

			\newenvironment{BoxRedText}
			{ 	\setlength{\fboxsep}{12pt}
				\begin{boxedminipage}[c]{1.0\linewidth}
				\color{red}
			}
			{ 	\end{boxedminipage} 
				\color{black}
			}
			
			

% ------------------------------------------------------------------------------
% Begin document (Content goes below)
% ------------------------------------------------------------------------------
	\begin{document}
	
			\dominitoc
			

			\title{지진}
			\author{김대희}
			\date{2017년 2월}
			\maketitle


			\tableofcontents
%			\listoffigures
%			\listoftables

			


	\clearpage
% ================================================= chapter 	====================
	\chapter{지진 일반}




	\clearpage
	% ------------------------------------------ section ------------ 
	\section{지진의 정의}

	지진은 한자어로 땅(地)지와 흔들릴 진(震)이 합쳐진 말로서 지구 내부의 에너지가 한 곳에 집중되어 있다가 한 순간에 돌출되어 나오면서 땅이 흔들이는 현상을 말한다. \\
		
	지진은 대부분 지구표면을 이루고 있는 여러 개의 판이 만나는 경계를 따라서 발생한다. \\
	
	\paragraph{자연재해}
	지진은 사람이 느끼지 못할 정도로 작은 경우도 있지만, 큰 지진이 발생하면 수많은 생명을 빼앗고 건물, 도로, 다리 등을 파괴하는 무서운 자연재해가 된다.
	
	지진은 홍수, 태풍 등과 힘께 대표적인 자연재해 중 하나이다. 
	과거부터 지금가지 지구에 살고 있는 우리 인류는 지진 재해로 인한 재난 혹은 재앙을 끊임 없이 겪어 왔고, 현재도 진행 중이다. \\
	
	지진과 같은 자연재해는 인명이나 재산에 피해를 줄 수 있는 자연현상을 말한다. \\
	
	\paragraph{재해}
	지진 현상 자체는 재해가 아니나 그것이 사람들에게 해를 끼칠 때 재해(Hazard)가 된다는 의미이다.
	
	\paragraph{재난}
	자연 재난은 보통 재해가 어떤 지점에서 한정된 시간에 걸쳐 발생하여 그것이 사회에 끼친 영향을 재난(isaster)이라고 한다.
	재난이란 용어는 자연현상이 사람과 관련되어 재산피해, 부상, 사망으로 이어졌을 때 주로 사용한다.

	\paragraph{재앙}
	재앙(Catastrophe)은 심각한 재난을 의미하며 재앙을 복구하는데 수많은 시간과 비용이 필요하다.
	지진재해로 인한 피해는 인명피해 이외에도 경제적인 손실과 실업, 정신적 피해, 생산성 감소 등의 사회적 손실을 들 수 있다. \\
	
	지진재난 혹은 재앙은 지구 내부의 힘으로 지각 판이 이동하고 충돌하는 과정에서 발생하는 지구상에 항상 존재하는 힘에 기이한다. \\
	
	
	\paragraph{재해의 피해결과}
	재해의 피해결과는 인구밀도와 토지이용도의 형태에 따라서 달라진다.
	지진재해 발생에 따른 인명피해 측면에서 예를 들면, 1985년 9월에 규모 8.0의 크기로 발생한 멕시코 시티 지진으로 10,000명 이상의 인명피해를 초래했다.
	1999년 터키 이즈미트 지진은 17,000명 이상의 목숨을 빼앗아 갔다.
	반면에 1995년 일본 고베에서 발생한 지진은 500명 정도의 인명피해가 초래되었다. \\
	
	인구와 자원이 밀집함에 따라 주기적으로 발생하는 지진재해에 피해도 증가하였다.
	이러한 경향은 지속되어 많은 사람들이 재해에 노출된 지역에 살고 있거나 주변 지역의 재해에 많은 영향을 받고 있다.
	사람들이 밀집되면 그만큼 더 위험에 놓이게 되고, 그들이 재해 지역으로 모이게 되므로 자연재난을 최소화시키는 계획이 더욱 필요하다.
	
	
	
	
	
	
	
	
	


	\clearpage
	% ------------------------------------------ section ------------ 
	\section{지진의 요소}

		지진을 이해하는데 있어서 지진요소를 빼놓을 수 없다.
		
		\paragraph{지진 요소}
		지진요소란 지진이 발생한 시각인 진원 시와 지진의 위치에 해당하는 진앙과 진원 깊이, 지진의 크기인 규모를 말한다.
		
		\paragraph{진원, 진원 깊이}
		여기서 진원이란 그림에서와 같이 지진이 시작된 곳을 말하고, 진원의 바로 위 지표면 지점을 진앙이라 하며, 지표에서 진원까지의 깊이를 진원 깊이라 한다.
		
		\paragraph{진앙}
		
		
		\paragraph{지진의 크기}
		지진의 크기는 규모와 진도로 설명한다.

		\paragraph{지진의 규모}
		규모는 지진 자체가 갖는 에너지의 크기를 말한다.
		따라서 지진파가 관측된 어느 곳에서 계산 하더라도 규모는 동일한 크기를 나타낸다.
		
		\paragraph{지진의 진도}
		반면에 진도는 지진파가 전달된 지점마다 다르게 표현된다.
		큰 지진이라도 아주 멀리서 관측되면 그 영향이 작아져 진도의 크기도 작아지고, 같은 지역에서도 지반조건이나 구조물의 노화 정도에 따라 진도가 달라질수 있다.
		
		\paragraph{지진의 규모와 진도}
		즉 그림 1.1과 같이 진원에서 일정 크기의 지진이 발생한 경우 진앙거리에 상관없이 모든 관측소에서 크기가 일정하게 측정되는 것을 ``규모''라 하고, 진원으로 부터 거리에 따라 그 크기가 다르게 나타나는 것을 ``진도''라고 한다.
		
		
	








	\clearpage
	% ------------------------------------------ section ------------ 
	\section{지진의 발생 원인}

	
	지구표면을 구성하는 암석을 지각이라 부르고, 이것은 커다란 퍼즐 조각처럼 몇 개의 조각으로 분할되어 맞물려져 있다.
	판이라고 하는 이 조각들은 연약권이라 불리는 점성을 가진 층 위를 연간 수 mm 이상의 속도로 천천히 움직인다.
	이에 관한 이론을 판구조이론이라 한다. \\
	
	서서히 움직이는 판은 서로 맞부딪치면서 암석들을 압축하거나 인장시켜 점차 내부의 압력을 증가시키게 된다.
	그 내부 압력이 매우 커져 더 이상 암석이 견디지 못할 때 암석의 연약 부분이 깨지고 갈라지면서 지진이 발생한다.
	이때의 충격으로 파가 발생하여 땅이 흔들리게 된다.
	
	
	\clearpage
	% ------------------------------------------ section ------------ 
	\section{국내의 지진 발생 현황}


	자료출처 : 국립 기상 연구원

	\clearpage
	% ------------------------------------------ section ------------ 
	\section{세계의 지진 발생 현황}

	자료출처 : 국립 기상 연구원



	\clearpage
	% ================================================= chapter 	====================
	\chapter{지진 방재 기본개념}




	\clearpage
	% ------------------------------------------ section ------------ 
	\section{지진 재해의 예측}


	\clearpage
	% ------------------------------------------ section ------------ 
	\section{위험 분석}



	\clearpage
	% ------------------------------------------ section ------------ 
	\section{지진 피해의 최소화}




	\clearpage
% ================================================= chapter 	====================
	\chapter{지진 발생과 지진파}


	\clearpage
	% ------------------------------------------ section ------------ 
	\section{지각 판의 변형}

		\paragraph{지각판} 
		세계에서 발생하는 지진은 특정 대상 지역에 집중하고 있기 때문에 지구 표면에는 변형이 축척되는 부분과 변형이 적은 부분이 존재한다.
		이런 변형이 적은 부분을 판(Plate)이러 부로고 있으며, 판에는 대륙판과 해양판 2종류가 있다.
		
		
		\paragraph{판구조론 Plate } 
		지구의 표층부에는 두께가 약 100km 인 몇개의 판으로 둘러싸여 있고, 
		이 판들이 그 하부 연약권에서 일어나는 맨틀의 대류에 의해 상대적으로 이동할 때 
		판의 경계부에서 화산활동, 지진활동, 조산운동 등의 지각변동이 일어난다는 이론이 판구조론이다. 
		
		\paragraph{베거너의 대륙이동설} 
		즉, 암석층으로부터 판이 지구 표면을 모자이크 형태로 덮으면서 횡방향으로 상대이동한다고 보는 것으로 
		그 기원은 1912년 독일의 알프레드 베게너\footnote{Alfred Lothar Wegener, 1880 $\sim$ 1930)} 가 제안한 대륙룩이동설로 거슬러 올라간다.
		또한 이 판을 움직이는 힘은 맨틀의 대류이다.
		
		\paragraph{유라시아판} 
		지구는 그림과 같은 지각판으로 이루어져 있다. 
		우리나라는 유라시아 판에 속해있고, 일본열도는 \textbf{유라시아판}과 \textbf{북미판} 위에 위치하고 있다.
		우리나라와 일본 열도는 지리적으로 가까이 있지만 지진발생 빈도 및 지진의 진도에 있어서 큰 차이가 난다.
		
		\paragraph{환태평양판} 
		일본은 태평양을 끼고 둥글게 분포하는 환태평양판에 해당한다. 
		태평양판과 필리핀판이 태평양 측의 해구 등에서 일본 열도 아래로 깊이 내려가 있는 형태이다.
		구체적으로 그림 에서 A-B 지점의 유라시아판과 태평앵판, 필리핀판에 이르는 지각 단면을 보면 그림과 같다.
		
		태평양판과 필리핀판이 일본열도 해양 지각을 따라 수렴하고 있어서 
		이 부분의 지각에 변형이 발생하면 지각 단면에 스트레스가 축적되고, 그 한계를 넘으면 단층이 파괴되면서 해구형 지진이 발생하게 되는 것이다.
		
		이러한 이유 때문에 일본에 지진이 자주 발생하고, 지진의 강도 또한 우리나라와 비교 할 수 없이 큰 지진이 발생하여 수많은 인명피해 뿐만 아니라 막대한 물적 피해가 발생하고 있다.
		
		
		

	\paragraph{일본의 동북부지진}
		2011년 3월에 일본 동북부 지역에서 발생한 지진은 규모 9.0으로 일본에서 기록된 지진 중 가장 큰 규모의 지진이다.
		그림 3.3과 같이 일본은 북미판과 태평양판, 유라시아판, 필리핀판이 만나는 경계로서 네 판이 만나는 경계부에서 서로 미는 힘이 발생한다. 
		이 판의 접촉면에서 큰 힘이 쌓이다가 견딜 수 없게 되면 순간적으로 태평양판의 위로 북미판이 올라서면서 쌓였던 에너지가 한꺼번에 방출되어 일본 동북부지진이 발생한 것이다.
		


	\clearpage
	% ------------------------------------------ section ------------ 
	\section{판의 경계와 지각변동}
	
	
	\subsection{발산경계, 생성경계}
		\paragraph{발산경계 Divergent Boundary} 
		고온의 맨틀 물질이 상승하면서 압력의 감소로 맨틀 물질이 부분 용해되어 마그마가 생성되고, 생성된 마그마가 지각의 틈을 통하여 분출한다.
		지각의 틈들은 주로 두께가 얇은 해양판에 주로 분포되어 있으며, 일직선상으로 연장 발달되어 있다.
		이곳에서 분출된 마그마가 해수에 의해 냉각되어 암석으로 굳어지면서 새로운 해양 지각을 만들고 새로운 해양지각은 계속되는 마그마의 분출에 의해 판 일부가 되어 점점 멀어지게 된다. 
		바로 이것이 판들이 멀어지는 발산경계이며 새로운 해양지각이 생성되는 곳이라 하여 생성경계\footnote{Constructive Boundary}라고도 한다.
		
		\paragraph{해령 Oceanic Ridge}  
		마그마의 분출에 의해 해저 산맥을 이루게 되는데 이를 해령이라 한다.
		\paragraph{열곡} 
		해령에서는 장력이 작용하여 해령의 중심부에서는 V자 계곡이 발달하고, 이 계곡을 지각 열류량이 높아 열곡이라 부르기도 한다.
		
	\clearpage	
	\subsection{수렴경계, 소멸경계}
		\paragraph{수렴경계 Convergent Boundary} 
		발산경계에서 두 개의 판이 서로 멀어지게 되면 판의 한 쪽에서 멀어지는 운동이 상대적으로 다른 쪽에서는 서로 가까워지게 되어 충돌하는 운동이다.
		이곳이 바로 판들이 만나는 수렴경계이며, 출돌에 의해 판들이 없어지는 곳이라 하여 소멸경계\footnote{Destructive Boundary}라고도 한다.
		
	
		\paragraph{해양판과 해양판의 수렴}
		해양판과 해양판이 만나는 경우 해령에서 생성된 해양판이 대륙에 접한 해양판 아래로 침강하여 들어가게 된다. 
		침강해 들어간 해양판이 대륙에 접한 해양판과 마찰에 의해 온도가 상승하고 해양 지각 일부가 녹아 마그마가 생성되고 분출하여 \textbf{호상열도}를 생성한다.
		한편 침강해 들어가는 해양판이 대륙에 접한 해양판과 마찰에 의해 끊임없이 지진이 발생한다.
		그 결과 해구 근처에는 천발지진이 발생하여 해구에서 멀어질수록 진원의 깊이가 점점 깊어진다.
		
	
		\paragraph{해양판과 대륙판의 수렴}	
		해양판과 대륙판이 만나는 경우 무거운 해양판이 가벼운 대륙판 아래로 침강하여 들어가게 된다. 
		해양판과 대륙판이 만나는 경계에서는 해구가 만들어지며 대륙 쪽에는 습곡산맥이 발달한다.
		침강해 들어간 해양판이 대륙판과의 마찰에 의해 온도가 상승하고, 해양 지각과 대륙 지각 일부가 녹아 마그마가 생성 분출하여 습곡산맥의 중심부에는 마그마가 관입한다.
		침강해 들어가는 해양판이 대륙판과 마찰에 의해 끊임없이 지진이 발생하고, 그 결과 해구 근처에는 천발지진이 발생하여 해구에서 멀어질수록 진원의 깊이가 점점 깊어진다. 
		로키 산맥과 안데스 산맥은 이러한 과정에 의해 생성되었다.
		
	
		\paragraph{대륙판과 대륙판의 수렴}
		두 대륙판이 만나는 경우 밀도가 비슷하여 두 대륙 사이에 쌓인 퇴적물과 두 대륙이 솟아올라 습곡산맥 같은 대규모의 조산대를 형성하게 된다.
		유라시아 대륙과 인도 대륙이 충돌하여 생성된 히말라야 산맥과 유라시아 대륙과 아프리카 대륙이 충돌한 알프스 산맥이 이에 속한다.
		
		
	
	\subsection{보존경계}
		판구조론에서 판의 경계는 변환단층, 해구, 해령으로 나누어진다.
		보존경계는 
	

	\subsection{단층의 변형}
	
	
	
	
	\clearpage
	% ------------------------------------------ section ------------ 
	\section{화산활동과 지진}

	이웃 나라인 일본은 지각변동이 활발하고 지진이나 화산 재해가 잦은 국가이다.
	이는 일본을 둘러싼 판 운동과 밀접한 관계가 있다.
	
	\paragraph{화산성 지진}

	대륙 지각판이나 해양 지각판이 어느 한 지각판으로 잠입함에 따라 
	지하 수백 km 부근에서 마그마가 생성되고, 상승하여 
	지표 부근에 마그마 집합소를 형성하여 지표의 회산으로 공급된다.
	
	
	화산에서는 마그마의 물리적 성질이나 화산활동 추이에 따라 단층에 의한 일반 지진 외에도 
	화산 특유의 다양한 형태의 지진이 발생한다. 이들을 총칭하여 화산성지진이라 한다.
	이는 보통 지진과는 달리 여진이나 전진이 없고 본진만이 발생하는 특징이 있다.
	
	\paragraph{화산성 지진의 발생}
	화산 주변 지하에는 마그마의 통로인 화구가 있다. 
	이 통로는 비교적 단단한 암석으로 이루어져 있으나 마그마가 상승하면 이 통로에 큰 압력이 걸리게 된다. 
	특히 지표 부근의 지하에는 지하수가 있어서 마그마의 열에 의해 가열된 물이 기화하여 체적이 수천 배 팽창하고 압력이 더욱 커진다. 
	그 압력으로 마그마 통로의 암반이 깨지고 그 파동에 의해 지진이 발생할 수 있다. 
	또한 마그마가 통과하고 난 뒤 압력이 낮아지면 그때까지 버티고 있던 암반이 붕괴되면서 지진이 발생히기도 한다.
	
	\paragraph{화산 분화의 예측}
	마그마는 화산의 분화와 관계없이 이동하지만, 이것이 곧 화산 분화와 직결되는 경우가 있다. 
	따라서 회산성지진의 관측에 의해 마그마의 이동을 포착하고, 화선성 지진의 패턴을 면밀히 분석해 분화 예측에 이용된다.
	

	\paragraph{백두산의 화산성 지진}
	화산성지진은 일반적으로 규모가 작고 사람이 체감할 수 없는 진도 1 이하가 많으며, 
	지진의 주기가 매우 다양하다는 특징이 있다. 
	우리나라에서는 백두산이 화산성 지진의 징후가 발견되어 지진학자들에 관심에 으르고 있다.
	
	화산성 지진은 분화구 아래 수십 km 이상의 깊이에서 발생하는 심부지진과 분화구 아래 수 km 이하의 얕은 곳에서 발생하는 천부지진으로 구분된다.
	
	심부지진은 지표 부근까지의 마그마 상승에 따라 발생한다고 보고 있으며, 천부지지은 4개의 형태로 분류된다.
	
	


	\clearpage
	% ------------------------------------------ section ------------ 
	\section{지진파의 성질}


	\clearpage
	% ------------------------------------------ section ------------ 
	\section{지진파의 전파}
	
	
	\clearpage
	% ------------------------------------------ section ------------ 
	\section{지진과 진자의 진동}
	
	
	
	\clearpage
	% ------------------------------------------ section ------------ 
	\section{진동과 충격}
	
	
	\clearpage
	% ------------------------------------------ section ------------ 
	\section{지진의 규모와 진도}
	
	
	
	
	
	
	
	
	
	
	
	
	\clearpage
	% ------------------------------------------ section ------------ 
	\section{진도계 - 지진동의 세기}
	
	
	\clearpage
	% ------------------------------------------ section ------------ 
	\section{지진계}
	
	
	\clearpage
	% ------------------------------------------ section ------------ 
	\section{칠레 지진}
	
	칠레 지진은 엄청난 재난임에 틀림없다.
	그러나 이 지진이 지진학의 발전에 크게 기여한 측면도 있다.
	특히 지진의 규모 때문에 지대한 관심을 모았다.\\
	
	\paragraph{지구의 자유진동}
	당시 이론적 연구에 따르면 대규모 지진은 지구 전체를 마치 거대한 종처럼 진동 시킬 것이라고 예측되었기 때문이다.
	이 현상을 \textbf{지구의 자유 진동}이라고 한다.
	
	대규모 지진이 아닌 경우에는 보통 \textbf{P파}나 \textbf{S파}가 또는 \textbf{표면파}가 발생한다.
	그러나 대규모 지진으로 지구 전체가 진동하게 되면 여러가지 형태의 \textbf{자유 진동}이 발생하며 
	그 주기는 매우 길어 수십분에 이른다. 
	이 \textbf{자유 진동}은 P파와 S파 및 표면파와 함계 지구 내부 구조를 분석하는 새로운 자료가 된다.
	따라서 칠레 지진이 지구의 자유 진동을 발생시켰는지가 비상한 관심사로 떠올랐다.
	
	\paragraph{캄차카 지진}
	1952년 11월 4일 캄차카에서 규모 9.0의 지진이 발생했을 때 베니오프 지진대를 발견한 휴고 베니오프\footnote{Hugo Benioff}가 자신이 만든 변형 지진계에서 주기가 매우 긴 지진파를 발견하고 그 중 주기가 대략 \textbf{57분}인 것을 \textbf{자유진동}이라고 보고 한 바 있다.
	그러나 그 후 7년간 유사한 현상이 관측되지 않아 많은 지진학자들은 베니오프의 기록을 오류라고 생각했다.
	
	\paragraph{칠레 지진}
	그러나 1960년 규모 9.5의 칠레 지진이 발생했을 때 미국, 일본, 유럽 등 세계 여러 곳에서 주기가 긴 자유 진동이 관찰되어 더 이상 이 현상을 의심할 수 없게 되었다.
	칠레 지진은 지구 내부를 연구할 수 있게 해 줄 새로운 수단을 지진학자들에세 선물한 사건이기도 하다.
	
	
	
	\clearpage
% ================================================= chapter 	====================
	\chapter{지진 발생 시 일어나는 특이현상들 }
	
	
	\clearpage
	% ------------------------------------------ section ------------ 
	\section{침강과 융기}

	% ------------------------------------------ section ------------ 
	\section{단층 운동}
	
	
	% ------------------------------------------ section ------------ 
	\section{산사태}


	\clearpage
	% ------------------------------------------ section ------------ 
	\section{액상화}

	지진으로 인해 격렬한 지반 진동이 발생하면 토양층의 수평방향의 진동에 대한 저항력, 즉 지구 물리학에서 말하는 전단 강도(shear strength)가 일시적으로 감소해 고체가 액체의 상태로 변하게 된다.
	이러한 현상을 액상화라고 한다.
	액상화현상이 발생하면 지면이 가라앉아 파열이 생기거나 또는 토사층이 유체처럼 낮은 곳으로 흘러내리게 된다.
	
	\clearpage
	% ------------------------------------------ section ------------ 
	\section{쓰나미, 지진 해일}
	
	쓰나미는 해저에서 지진이 발생해 해저 지층이 드러스트 단층에서 수직 운동이 일어날때 발생한다.
	이 진동으로 흔들린 바닷물의 파동은 사방으로 전파되어 나가다가 근처 및 먼곳의 해안을 만나면 높은 파도와 해일을 일으키며 육지로 범람한다.
	 쓰나미는 해저 수심의 제곱근에 비례하는 속도로 전파된다.
	 평균 수심이 5킬로미터 정도인 태평양에서 쓰나미가 일어날 경우 시속 800킬로미터로 달린다.
	 그 주기는 대략 1시간이고 파장을 대략 800킬로미터가 된다.
	 수심이 깊은 먼바다에서 쓰나미의 파고는 수 센티미터에 불과하다. 따라서 1시간 주기로 진동하는 쓰나미의 해일을 진앙 근처를 항해하던 배라고 하더라도 감지하기는 어렵다.
	 이런 까닭으로 쓰나미는 결코 바다에서 관측되지 않는 것이다. 
	 그러나 쓰나미가 해안선에 접근하면 수심이 감소하고 그에 따라 쓰나미의 전파 속도도 감소하며 에너지 보존의 법칙에 따라 그 운동에너지를 유지하기 위해 파고가 증가한다. 
	 때로는 높이가 수십 미터에 이르는 해수벽(water wall)이 형성되어 저지대의 해안 지역을 덮쳐 엄청난 피해를 입힌다.
	 쓰나미는 보통 지진 규모가 7.5 이상이어야 한다.
	 
	 \paragraph{동일본 대지진}
	 2011년 3월의 동일본 대지진은 1896년 산리쿠 지진의 진앙에서 동남 방향으로 대략 200킬로미터 떨어진 시점에서 일어났고, 이 지니으로 최대 파고 40.5미터에 이르는 대규모 쓰나미가 발생해 일본 동북 해안 지역을 덮쳐 막대한 피해를 입혔고 수만명이 사망$\sim$실종 되었다.
	 이 쓰나미는 태평양을 건너 알래스카에서 칠레까지 남북아메리카의 전 해안에 이르렀으나 그 피해는 경미했다.\\
	 
	특히 동일본 대지진에서 발생한 쓰나미는 후쿠시마 원자력 발전소의 냉각수 공급 시스템을 파괴했고, 과열된 원자로가 폭발해 방사성 물질이 누출되었다.
	결국 발전소 주변 반지름 30킬로미터 안의 주민 수십만 명이 이주하는 등 엄청난 재난이 발생했다.
	발전소에서 대기 중으로 방출된 방사성 물질은 이 일대의 광범위한 지역을 사람이 살수 없게 만들었고 바다로 방출된 오염수는 국제적 분규를 일으켰다.
	
	 
 



	\clearpage
	% ------------------------------------------ section ------------ 
	\section{세이슈 seiche}
	

	% ------------------------------------------ section ------------ 
	\section{지면파}

	% ------------------------------------------ section ------------ 
	\section{지진광}
	
	
	
	\clearpage
% ================================================= chapter 	====================
	\chapter{지진 발생 매커니즘}
	


	\clearpage
	% ------------------------------------------ section ------------ 
	\section{고대의 지진 발생에 대한 인식}

		19세기까지 지진은 관찰할 수 없는 깊은 곳에서 발생하기 때문에 그 원인을 알수 없었다.\\
		
		고대로부터 사람들은 어떤 괴물이 지구를 받치고 있다고 생각하고 그것이 몸을 흔들 때 지진이 발생한다고 설명했다.
		\paragraph{메기, 고래, 거북}
		일본에서는 그 괴물이 거대한 메기였고, 남아메리카의 일부 지역에서는 고래였고, 북아메리카의 원주민들은 거대한 거북이라고 생각했다.
		
		\paragraph{개구리}
		몽고의 라마승들은 신이 지구를 만든 후 거대한 개구리 등에 얹어놓았는데, 개구리가 머리를 흔들거나 발을 뻗을 때, 바로 그 위에서 지진이 발생한다고 설명했다.
		
		\paragraph{아리스토텔레스}
		고대 그리스의 철학자 아리스토텔레스는 지진은 지구 내부의 공동에 있는 공기나 가스가 밖으로 빠져 나오려고 바둥거릴 때 발생한다고 생각했다.
		그리고 이런 일이 일어나기 전에는 먼저 공동 속으로 바람이 불러 들어가야 하기 때문에 지진이 발생하기 전에 날씨가 숨 막을듯이 답답해진다고 생각했다.
		이것을 지진 날씨(earthquake weather)라고 했다.

		\paragraph{인간의 죄악에 대한 신의 응징}
		그러나 서양에서 오랜 기간 가장 유력한 것으로 군립했던 견해는 지진이 사람들의 죄악에 대한 신의 응징으로 발생한다는 것이었다. \\
		
		이러한 생각들은 기체의 성질이나 지각의 구조, 그리고 지진이 발생하는 깊이 등이 밝혀지고 과학적 지진학이 등장하자 바로 폐기되었다.
		

	\clearpage
	% ------------------------------------------ section ------------ 
	\section{탄성 반발력 이론}
	
	그렇다면 지진은 왜 발생하는 것일까? \\
	
		\textbf{대부분의 지진들은 지구 내부에 작용해 지각을 변형시키는 
		지구구조력(geotectonic force) 또는 구조력(tectonic force)에 의해 단층에서 지층이 깨어지면서 발생한다.} \\
	
		지진 발생의 매커니즘에 관한 이러한 이론을 \textbf{탄성 반발설}(elastic rebound theiry)이라 하고 
		미국 존스 홉킨스 대학교의 해리 라이드 교수에 의해 제창되었다.

		\paragraph{해리 라이드의 샌앤드리어스 단층 분석}
		1906년 샌프란시스코 지진이 발생하기 이전에 
		다행히 이 지진으로 인해 깨어진 샌앤드리어스 단층의 일부에 대한 측지 작업이 두 번 수행되었다.
		해리 라이드는 이 측지 데이터와 지진 발생 후 획득한 데이터를 종합해 지진 발생과 연관된 지각 변형을 분석했다. \\
		
		라이드는 단층 주위의 암석이 오랜 기간 응력을 받아 변형되면서 
		마침내 그 변형을 지탱할 수 없는 한계에 이르면 
		가장 약한 부분이나 변형이 가장 큰 부분이 단층면을 따라 순간적으로 깨지면서 응력이 방출되고 
		탄성 반발에 의해 영구 이동, 즉 오프셋이 생긴다고 생각했다.
		이때 깨어진 부분에서 지진이 발생하며 
		단층 주위의 변형된 암석에 모여 있던 \textbf{탄성에너지}의 일부가 
		\textbf{파동 에너지}로 변해 사방으로 퍼져 나간다.
		이것이 지진이 되는 것이다.
		이러한 지진 원인 설명을 ``\textbf{탄성 반발설}''이라고 한다. \\
		
		이전에 단층은 지진으로 인해서 생긴다고 생각했으나 
		탄성 반발설은 지진이 바로 단층 운동으로 인해 발생한다고 주장한것이다.
		탄성반발설은 지각에서 발생하는 모든 지진이 단층에서 발생함을  보여준 것으로 
		20세기 지진학의 가장 위대한 발견이라고 볼 수 있다.
		
				
		
	\clearpage
	% ------------------------------------------ section ------------ 
	\section{단층}

		\paragraph{탄성변형 elastic deformation}
		암석이 응력을 받으면 그 응력에 비례하는 변형이 일어나고 동시에 변형을 없애려는 복원력이 작용한다.
		응력을 제거하면 복원력에 의해 변형이 사라져 다시 원상태로 돌아온다.
		이러한 변형을 탄성 변형(elastic deformation)이라 한다.
		
		\paragraph{탄성 elastic}
		탄성은 외부 힘으로 인해 변형을 일으킨 물체가 힘이 제거 되었을 때 원래의 모양으로 되돌아가려는 성질로 
		일상에서는 고무나 스프링 등에서 쉽게 볼 수 있다. 
		
		\paragraph{소성 plasticity}
		그런데 응력이 증가해 어떤 한계점(탄성 한계점)을 초과하면 \textbf{응력을 제거해도 변형이 없어지지 않는다}.
		이렇게 고체에 외부 힘이 작용해 탄성 한계 이상으로 변형시켰을 때, 
		외력을 빼어도 원래의 상태로 돌아가지 않는 성질을 소성(소성, plasticity)라고 한다. 
		
		\paragraph{소성 변형 plastic deformation}
		이 탄성 한계점 이후의 변형을 소성 변형(plastic deformation)이라 한다.
	
		\paragraph{단층}
		탄성 한계점에서 일부 암석은 깨진다.
		그리고 응력이 증가하면 소성 변형을 하던 암석들도 결국 깨지고 만다. 
		암석들은 지구 표면에서는 단단해 잘 깨지지만 지구 내부의 높은 온도와 압력에서는 잘 깨지지 않는다.
		그러나 매우 강한 응력이 작용하면 결국은  깨지게 된다.
		이때 깨진 면을 경계로 지층이 엇갈려 이동하게 되고 이것을 단층(fault)이라 한다. 
	
		\paragraph{단층은 지구조적운동의 표식}
		단층은 지질 구조의 불연속면에서 확인 할 수 있으며, 
		단층은 과거 어느 시점에서 단층면을 따라 \textbf{지구조적 운동}이 일어났음을 가리키는 표식이 된다.
		그리고 단층의 길이는 1미터 미만에서 수백 킬로미터의 넓은 범위에 걸친다.
			
		\paragraph{주향과 경사}
		지층과 같은 평면 지질 구조의 방위를 표시하는 데 주향(strike)과 경사(dip)가 사용된다.
		주향은 지층과 수평면이 만나서 이루는 직선의 방향을 말한다.
		경사는 지층에서 주향에 수직한 방향이 수평면과 이루는 각을 말한다.
		
		\paragraph{단층의 분류 : 정단측, 역단층, 주향 이동 단층}
		단층의 대표적인 유형으로는 정단층, 역단층, 주향이동단층의 세 가지가 있다.
		단층의 유형은 단층이 생길 때 어떤 힘이 작용했느냐에 따라 달라진다.
		
		\paragraph{지구력의 분류 : 인장력, 압축력, 전단력}
		지각의 변형을 일으키는 힘을 지구조력이라고 하는데, 
		지구조력에는 압축력(compressive force), 장력(tensile force)과 전단력(shearing force)의 세 종류가 있다.

		\paragraph{압축력에 의한 역단층}
		어떤 물체의 크기를 작게 하려는 작용이 있을 수 있다.
		이때 작용하는 힘을 압축력이라 한다.
		압축력이 작용하면 지각이 짧아지며 휘어져 습곡(fold)이 일어나다가 마침내 깨어지며 역단층이 발생한다.

		역단층에서 단층면의 경사가 45도 이하인 것을 드러스트 단층(thrust fault) 또는 충상 단층이라 한다.
		
		\paragraph{인장력에 의한 정단층}
		장력이 작용하면 지각이 늘어나 얇아지다가 깨어어지며 정단층이 발생한다.
		장력은 물체의 표면을 끌어당겨 그 크기를 증가시키는 힘을 말한다.
			
		\paragraph{경사 이동 단층}
		정단층과 역단층처럼 경사면을 따라 수직적인 변위가 일어나는 단층을 경사 이동 단층(dip-slip fault)이라 한다.
		
		\paragraph{전단력에 의한 주향 이동 단층}
		물체 내의 한 면을 따라 그 양쪽에서 서로 같은 크기로 반대 방향으로 작용하는 힘을 전단력이라 한다. 
		전단력이 작용하면 지각이 한면을 경계로 서로 반대 방향으로 움직이다 깨어지며 주향 이동 단층을 이룬다.

		\paragraph{우수 주향 이동 단층, 좌수 주향 이동 단층}
		주향이동단층에서 단층의 한쪽에 서서 볼때 다른 쪽이 오른쪽으로 이동하면 우수 주향 이동 단층(right-lateral strike-slip fault)이라 하고, 왼쪽으로 이동하면 좌수 주향 이동 단층(left-lateral strike-slip fault)이라 부른다.
		
		\paragraph{사교 단층 oblique-slip fault}
		장력 또는 압축력과 전단력이 함께 작용하면 사교 단층이 생긴다.		
		\textbf{경사이동단층}에서는 지층이 수직 방향으로만 변화하지만 
		\textbf{주향이동단층}에서는 오직 수평방향으로만 변화한다.
		실제로 발생하는 단층은 경사이동단층이나 주향단층이 아니라 이 단층들이 혼합된 사교 단층의 형태 나타난다.
		
		
		
		\paragraph{단층 크리프}
		어떤 단층에서는 지진이 발생하지 않고 
		거의 연속적이거나 아니면 아주 작은 크기의 불연속적인 운동이 일어나는 경우가 있다. 
		이러한 종류의 운동을 단층 크리프(fault creep)라고 한다.
		크리프는 ``조금씩 미끄러져 나가다''라는 뜻인데 적절한 번역어가 없어 영어 단어를 음역해 쓰고 있다.

		\paragraph{단층 크리프의 발생 매커니즘}
		이 단층 크리프의 발생 매커니즘은 이렇다.
		오랜 기간 암석들이 단층면을 따라 깨지며 이동할 때 작은 가루로 부서진다.
		이 암석 가루들이 단층 주위에서 침투하는 지하수와 만나면 \textbf{단층 점토}(fault clay)를 생성한다.
		이 단층 점토는 응력에 대해 소성 변형을 일으켜 지진 대신 단층 크리프가 발생한다.\\

		\paragraph{단층 점토}
		샌앤드리어스 단층과 같은 대규모 단층대에서는 
		지표면의 연약한 토양층 하부에 \textbf{단층 점토}가 수 킬로미터 지하까지 이어진다.
		더 아래로 내려갈수록 점점 더 단단해지는 암석층을 만날 수 있다.\\
		
		\paragraph{지진의 발생 위치 : 단층 점토 하부에서 발생 }
		지진은 탄성 변형을 통해 에너지가 축적될 수 있는 단층 점토 하부의 암석층에서 발생한다. 
		더 깊이 내려가면 지구 내부의 온도가 상승해 
		다시 암석이 응력에 대해 소성 변형을 하게 되어 지진이 잘 발생하지 않게 된다.\\
		
		지진이 많이 일어나는 캘리포니아 중부의 대부분 지역에서도 
		지진들은 15킬로미터 이하의 깊이에서는 발생하지 않는다.
		
		\paragraph{활성 단층 active fault}
		단층은 지각의 약한 부분이므로 
		지구조력이 지속적으로 작용하면 결국 여기에서 지층이 깨지며 지진이 발생하게 된다.
		이처럼 지진이 발생하는 단층을 활성 단층이라 한다.
		
		\paragraph{비활성 단층 inactive fault}
		단층이라고 해서 모두 활성 단층인 것은 아니다.
		지표면에 드러나 있는 대부분의 단층에서 지진이 발생하지는 않는다.
		국지적으로 작용하는 응력이 오래전에 사라졌거나 
		아니면 지하수의 침투로 화학 작용이 일어나 파열면이 아물어 들었기 때문이다.
		이러한 단층에서는 더 이상 지진들이 발생하지 않게 되며 이러한 단층을 비활성 단층이라 한다.
		
		\paragraph{활성 단층과 비활성 단층의 구분}
		지질학적으로 어떤 단층에서 제4기(250만년 전부터 현재까지)에 지진이 발생하지 않았으면 
		그 단층을 비활성단층으로 간주한다.
		제4기에 지진이 발생한 지질학적 증거가 나타나면 현재 지진이 발생하지 않고 있다 해도 앞으로 지진이 발생할 가능성이 있는 활성단층으로 간주한다.
		따라서 지진 재해의 측면에서 활성 단층의 형태는 매우 중요하다.
		

		\paragraph{활성 단층의 조사 - 고지진학}
		활성단층은 지형 조사를 통해 확인할 수 있다. 수천 년에 걸쳐 간헐적으로 발생하는 단층 운동은 함몰 못(sag pond), 한 줄로 늘어선 우물들, 신선한 단층 절벽(fault scarp) 같은 것들을 지형 위에 흔적으로서 남긴다.
		그러나 이러한 지형 변형들이 일어난 순서와 그 시기를 결정하는 것은 쉬운 일이 아니다.
		단층을 덮고 있는 제4기 토양이나 퇴적층을 `도랑 파기' 또는 트랜치(trench)'를 해서 \textbf{지층들의 오프셋을 확인}할 수 있다면, 이 지층들이 생성된 연대를 측정해 그 단층에서 과거에 발생한 지진들의 역사를 밝힐 수 있다.
		지층의 생성 연대는 그 지충에 묻혀 있는 나뭇잎이나 가지 등 유기물질에 포함된 방사상 물질의 양을 측정함으로써 밝힐 수 있다.
		
		이러한 방법으로 제4기 지층에서 확인된 지진들을 고지진이라 하고 이런 지진들을 연구하는 분야를 고지진학학이라 한다. 고지진학의 연구는 1975년 스탠퍼드 대학교의 대학원 학생인 캐리 시(Kerry Sieh)에 의해 시작되었고 현재 활성 단층을 연구하는 중요한 수단으로 활용되고 있다.
		
		\paragraph{지구조 지진 - 텍토닉 지진}
		지구 내부에서 작용해 지각 변형을 일으키는 거대한 지구력으로 인해 발생하는 지진을 지구조 지진(tectonic earthquake, 텍토닉 지진)이라 한다.
		지구에서 발생하는 대부분의 지진들이 지구조 지진에 속한다.
		지구조 지진은 지구 내부 구조를 이해하기 충분한 에너지의 지진파를 발생시켜 과학적으로 매우 중요하다.
		뿐만 아니라 엄청난 지진 재해를 불러올 수 있으므로 지진 재해의 측면에서도 심각한 위협이 된다. \\



	\clearpage
	% ------------------------------------------ section ------------ 
	\section{화산 지진}
		
		지구조 지진 외에도 몇가지 다른 종류의 지진들이 있다.

		\paragraph{화산 지진 volcanic earthquake}
		화산 활동이 종종 지진을 일으키기도 한다.
		이러한 지진을 화산 지진이라 한다.\\
		
		
		1959년 11월 중순 하와이의 킬라우에아 화산이 크게 분출했다.
		8월 초부터 현지 화산 관측소의 지진계에 심부 마그마의 운동과 연관된 것처럼 생각되는 지진들이 
		대략 50킬로미터 깊이에서 발생하는 것이 기록되었다.
		9월 중순에는 이 지진들이 800미터 깊이까지 상승해 용암이 지표 가까이 이르렀음을 알 수 있었다.
		또 지진의 발생 빈도가 급증해 두 달 동안에 2만 2000회의 지진들이 발생했다.
		그러나 이 지진들은 모두 사람들이 감지하기에는 너무 작은 지진들이었다. 
		
		화산지진들은 화도 하부에서 고온의 마그마가 상승함에 따라 주위 암석이 응력을 받아 변형 되다가 깨어지면서 발생한다.
		때로는 화산 주위의 단층이 깨어지며 지진이 발생하고 지진파가 화산 하부의 마그마를 자극해 용암이 분출하기도 한다. 
		
		1975년 11월 29일 오전 5시경 킬라우에아 화산 근처에서 강력한 지진이 발생했다. 
		이 지진과 연관된 해저 단층 운동으로 쓰나미가 발생해 인명 피해가 생겼다. 
		지진 발생 후 약 1 시간 후에 킬라우에아 화산에서 대규모로 용암이 분출했다.
		

	\clearpage
	% ------------------------------------------ section ------------ 
	\section{산사태로 인한 지진}

		\paragraph{산사태로 인한 지진 발생}
		대규모의 산사태로 지진이 발생하기도 한다.
		1974년 4월 25일 페루의 만타로 강을 따라서 발생한 대규모 산사태는 규모 4.5에 해당하는 지진을 발생시켰고 450명이 사망했다.
		이 지진은 암석과 토양이 중력으로 인해 급격이 하강하면서 중력 에너지의 일부가 지진파 에너지로 바뀌어 발생한 것이다.
		
	\clearpage
	% ------------------------------------------ section ------------ 
	\section{인공 지진}
		
		\paragraph{인공 지진}
		대부분의 지진이 자연적인 원인으로 발생하지만 때로는 인간의 활동이 지진을 일으키기도 한다.
		이러한 지진들을 인공 지진 또는 유도 지진이라 한다.

		\paragraph{인공 지진의 발생}
		인공 지진은 지하에서 화약이나 핵폭탄을 폭발시키거나, 
		댐을 쌓고 물을 저수하거나, 우물을 파고 지각에 물을 유입하거나, 
		지하 동굴이나 광산의 천장이 무너질 때 발생한다.

		\paragraph{폭파 지진}
		인공 지진 중에서 지진학 발전에 크게 기여한 것이 
		지하에서 화약이나 핵폭탄을 폭발시킬 때 일어나는 폭파 지진이다.
		폭파 지진은 진앙과 발파 시점을 정확히 알 수 있어, 
		\textbf{진앙}과 \textbf{발생 시간}을 추정해야 하는 \textbf{지구조 지진} 보다 
		지구 내부 결정에 더 좋은 자료를 지진학자들에게 제공해 준다.
		지난 수십 년에  걸쳐 세계 여러 지역에서 실행한 핵폭발 실험으로 상당한 규모의 폭파 지진들이 발생했다.
		이 지진들의 에너지는 충분히 커서 지구조 지진에서 얻은 자료를 보완해 지구 내부 구조를 규명하는데 기여를 했다.
		지하 시추공에서 핵폭발이 일어나면 수백만분의 1초 동안에 수천기압의 압력과 수백만 도의 열이 발생해 주위 암석을 기체로 바꾸고 지하에 공동을 만들게 된다. 이 공동이 점차 커지다가 주위 암석이 결국 폭파의 충격으로 깨지면 지진이 발생한다.
		1968년 4월 26일 미국 네바다 핵무기 실험장에서 대규모 핵폭발 실험을 할 때 이로부터 50킬로미터 떨어진 라베이거스의 시민들은 이 실험으로 건물이 파괴되고 심지어 사망자까지 발생할지 모른다고 걱정했다.
		그러나 핵실험은 강행되었고 건물들은 흔들렸으나 별다른 피해는 없었다.
		이 폭발은 캘리포니아 대학교 버클리 캠퍼스의 지진계에 규모 6.4의 지진으로 나타났다.
		인공 지진이나 유도 지진은 대부분 규모가 작다.



	\clearpage
	% ------------------------------------------ section ------------ 
	\section{단층면해}
		
		\paragraph{진원, 진앙}
		지진들은 지각에서 단층이 깨어지면서 움직일 때 발생하나 대부분의 경우 깨어진 부분이 지표에 이르지 못한다.
		이 경우 단층이 깨어지기 시작하는 지점을 진원(focus)이라 하고 
		그 수직 상부에 있는 지표면의 지점을 진앙(epicenter)이라 한다.
		
		지진파는 진원으로부터 사방으로 전파한다.
		지진이 발생할 때 단층이 동시에 깨지는 것이 아니고 진원에서 시작된 파열이 단층면을 따라 S파 속도의 70$\sim$90 퍼센트의 속도로 전파하며 단층이 점차 깨어지게 된다.
		응력으로 인해 진원 주위에 축적된 변형이 소멸될 때까지 파열이 진행되며 이 과정에서 지진파가 발생한다.
		이 경우 파열된 단층면 전체가 지진원(earthquake source)이 된다.
		
		
		
		
		\paragraph{단층면해}

		지구 내부의 구조와 아울러 지구 내부에 작용해 지진을 일으키는 지구조력 및 국지적 응력의 공간적 분포를 
		아는 것은 지진학의 매우 중요한 과제이다.
		
		대부분의 경우 단층면을 지상에서 육안으로 확인하기는 힘들다.
		왜냐하면 지하에서 발생한 파열면이 지표까지 연장되지 않기 때문이다. 
		그러나 지진이 일어날 때 발생하는 지진파를 분석하면 단층 운동과 작용하는 응력의 분포를 알아낼 수 있다.
		
		지진이 발생하면 진원으로 부터 지진파가 사방으로 전파하며 P파가 처음 관측소에 도달하고 S파와 표면파가 그 뒤를 따른다.
		이 경우 P파에 의한 지면의 첫 진동을 초동(first motion)이라 한다.
		 지진을 일으키는 단층 운동을 지진 발생 매커니즘 또는 진원 메커니즘이라 한다.
		 지진학자들은 이 매커니즘을 탄성 반발 모형으로 설명한다.
		 지진학자들은 지구 표면 여러 곳에서 관측되는 지진파를 분석해 단층 운동의 방위와 유형(정단층, 역단층, 주향 이동단층)과 지진을 일으킨 응력(인장력, 압축력, 전단력)의 분포를 알아낼 수 있다. 
		 
		 
		 지진 발생 매커니즘에 대한 이러한 해석을 단층면해(fault plane solution)라 한다.
		 
		지구에서 발생하는 지진들의 단층면해는 판구조론의 가설을 이론으로 정립하는 데 결정적인 기여를 하게 되었다.
		판구조론의 가설에 따른면 판들이 서로 멀어지는 경계에서는 정단층, 충돌하는 경계에서는 역단층, 그리고 서로 엇갈리는 데서는 주향 이동 단층이 발생할 것으로 예상되는데 실제로 이 경계들에서 발생는 지진들의 발생 메커니즘은 판구조론의 이론과 부합함이 밝혀졌다.
		
		
		



	\clearpage
% ================================================= chapter 	====================
	\chapter{지진과 판구조론}




	\clearpage
% ================================================= chapter 	====================
	\chapter{지진계와 지진의 관측}


	\clearpage
	% ------------------------------------------ section ------------ 
	\section{지진파 - 지진파는 지구 내부의 비밀을 밝히는 열쇠}

	지각의 단층이 깨어지며 지진이 발생하게 되면 진원으로부터 두 종류의 파동이 발생해 지구 내부를 통과하게 된다.
	이 파동들은 20세기에 지진계가 발명되기 전에 19세기에 프랑스의 수학자 시메옹드니 푸아송이 그 존재를 펜과 종이를 사용해 이론적으로 밝혔다. \\
	
	푸아송에 따르면 탄성체인 고체 내부에 두 가지 종류의 변형이 일어나고 그것들은 각기 다른 속도의 파동으로 탄성체 내부로 전파해 간다.
	
		\paragraph{압축파 compressional wave P파}	
		첫째 종류의 변형은 파동이 전파하는 방향으로 일어나 탄성체의 체적을 변화시키는데(따라서 밀도가 변한다.) 이 지진파를 압축파라 한다. 압축파는 대기의 밀도가 변하면서 전파는 되는 음파, 즉 소리의 파동과 비슷하다.

		\paragraph{전단파 shear wave S파}	
		둘째 종류의 변형은 체적(또는 밀도)의 변화 없이 탄성체의 모양이 파동이 전파하는 방향에 수직으로 비틀어지는 것이며 이 변형을 전달하는 지진파를 전단파라 한다.
		
		\paragraph{P파, S파}	
		그런데 압축파의 속도가 전단파의 속도보다 빨라 압축파가 먼저 도착하는 파동이라 해서 P파 (Primary wave)라 하고 전단파는 다음에 도착하는 파동이라 해서 S파 (Secondary wave)라 부른다.


		\paragraph{압축파의 움직임}	
		P파가 전파할 때 매질의 한 점은 전파 방향으로 진동함으로써 매질의 체적(또는 밀도)의 변화가 일어난다.
		P파가 도달하기 이전의 정사각형은 전파방향으로 확장 또는 축소된다.


		\paragraph{전단파의 움직임}	
		S파가 전파할때 매질의 한점은 전파 방향에 수직으로 진동하며 도달하기 이전에 정사각형은 평행사변형으로 변했다 다시 정사각형으로 되돌아 온다. 체적의 변화는 일어나지 않으므로 밀도의 변화도 발생하지 않는다.

		\paragraph{실체파 body wave}	
		지각에서 P파의 속도는 대략 초속 6킬로미터이고 소리보다 대략 20배 빠른다. 
		S파는 대략 3.5킬로미터로 지각을 전파한다. 
		P파와 S파가 모두 지구 내부를 통과해 전파하는데 이런 이유에서 이들을 실체파(body wave)라 부른다.
		


		\paragraph{표면파 surface wave}	
		실체파인 P파와 S파가 지구 표면에 이르면 밖으로 전파하지 않고 반사되어 다시 지구 내부로 되돌아온다.
		이때 표면으로 올라가는 파와 표면에서 반사되어 되돌아오는 파들이 서료 겹쳐 간섭해 전혀 다른 종류의 파동인 표면파(surface wave)를 생성하게 된다. 
		표면파는 지구 표면을 따라 전파하며 그 진폭이 지구 내부로 들어가면서 급격히 감소하는 특성을 지닌다.
		
		표면파의 종류에는 레일리파와 러브파가 있으며 각기 19세기 말에 영국의 물리학자 레일리 경, 즉 존 월리엄 스트럿과 20세기 초 영국의 수학자 오거스터스 러브가 이론적으로 그 존재를 밝혔다.
		표면파는 호수의 표면에 이는 잔물결과 같다.
		표면파는 실체파보다 더 느린 속도로 전파되며 러브파가 레일리파보다 더 빠르다.
		
		\paragraph{레일리파}
		레일리파가 전파할 때 지면은 수직으로 타원 운동을 하며 그 운동 방향은 전파 방향과 반대이다 
			
		\paragraph{러브파}
		러브파의 경우 지면은 S파와 같이 전파 방향에 직각으로 우직 수평으로만 진동한다.
		

		\paragraph{레일리 경}
		레일리 경우은 남작가문의 세 번째 당주였으며 아버지가 죽자 작위를 이어받았다.
		그는 전자기 이론을 완성한 제임스 맥스웰의 뒤를 이어 케임브리지 대학교의 케번디시 물리학 석좌 교수가 되었으며 아르곤을 발견해 1904년 노벨상을 수상했다. 그는 하늘이 왜 파랗게 보이는가를 산란 이론으로 설명했다. 빛의 파장보다 작은 입자, 예를 들어 공기 중 기체 분자 등에 의해 빛이 산란되는 것을 레일리 산란 이론이라 한다.
		
		
		
		
		
		
	
	
		
	

















	\clearpage
	% ------------------------------------------ section ------------ 
	\section{지진계 seismograph}

		지진이 발생하면 \textbf{진원}으로 부터 실체파와 표면파가 각기 지구 내부와 표면을 전파해 멀리 떨어진 지점에 도달한다. 지진에 의한 지면의 진동을 기록하는 기계를 지진계라 한다.
		지진계에 속도가 가장 빠른 P파가 먼저 기록되고 그 다음에 S파와 표면파의 순서로 기록된다.
		실체파는 그 파동에너지가 3차원의 공간으로 확산함에 비해 표면파는 2차원의 공간으로 확산됨으로 거리에 따른 에너지의 감소는 표면파가 실체파보다 작아 진앙에서 먼 지점에서는 보통 표면파의 진폭이 실체파보다 크게 나타난다.
		
		지진계는 지진학자들로 하여금 접근할 수 없는 지구 내부에서 발생하는 지진들을 관측하고 분석할 수 있게 해 준 놀라운 발명품이다.
		
		\paragraph{중국 장현의 감진기 seismoscope}
		지진을 기록하는 최초의 장치는 기원전 132년에 중국의 \textbf{장형}이 만들었는데  지름이 2미터 정도인 술병과 같은 모양으로 그 둘레에 8개의 용머리가 있고 그 아래에 입을 벌리고 있는 두꺼비들이 있었다. 각 용의 입에는 청동 공이 있었고 지진이 발생하면 그 중 하나가 금속 두꺼비 입속으로 떨어지게 되어 있었다. 장형은 만약 남쪽에 있는 두꺼비가 공을 잡으면 지진이 북쪽에서 발생한 것이라고 생각했다. 그의 기구는 단지 지진의 발생과 그 방향만을 기록하는 것이었으모 이를 지진계와 구분해서 감진기라 부른다. 장형의 감진기를 지동의라 부르기도 한다.
		
		
		
		
		
		


	\clearpage
	% ------------------------------------------ section ------------ 
	\section{지진 관측망}




























	\clearpage
% ================================================= chapter 	====================
	\chapter{내진설계}


	
	
	
	
	



	

% ------------------------------------------------------------------------------
% End document
% ------------------------------------------------------------------------------
\end{document}


